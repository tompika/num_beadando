\documentclass[12pt]{extarticle}
\usepackage[utf8]{inputenc}
\usepackage{amsmath}
\usepackage{verbatim}
\usepackage{setspace}
\onehalfspace


\begin{document}

\centerline {\bf Belső szorzat}
\noindent
Két vektor belső szorzatának kiszámítása.

\noindent
{\bf Input}
\begin{flalign*}
&d&\\
&v_{1}\ldots v_{d}\\
&w_{1}\ldots w_{d}
\end{flalign*}


\noindent
{\bf Output}
\begin{flalign*}
& \sum_{k=1}^d v_{k}w_{k}&
\end{flalign*}


\noindent
{\bf Korlátok}\newline
$0<d<100$  A kiírt számok {\bf 12} értékes jegyet tartalmazzanak!


\noindent
{\bf PéldaInput}
\verbatiminput{../io/in4}

\noindent
{\bf PéldaOutput}
\verbatiminput{../io/out4}

\end{document}
