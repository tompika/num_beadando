\documentclass[12pt]{extarticle}
\usepackage[utf8]{inputenc}
\usepackage{amsmath}
\usepackage{verbatim}
\usepackage{setspace}
\onehalfspace


\begin{document}

\centerline{\bf Newton-módszer polinomra }
\noindent Közelítsük egy $n$ fokú $p$ polinom gyökét, $x_0$ kezdőpontból $m$ lépést végezve! 
Ha $x_m=\pm\infty,\texttt{NaN}$ akkor a \texttt{fail} sztringet írjuk ki!


\noindent
{\bf Input}
\begin{flalign*}
& n\: x_0\: m  &\\
& p_n \ldots p_1 \: p_0\\
& 
\end{flalign*}


\noindent
{\bf Output}
\begin{flalign*}
& x_m &
\end{flalign*}


\noindent
{\bf Korlátok}\newline
$0<m,n<100.$ A kiértékeléseket Horner-módszerrel végezzük!
A kiírt számok {\bf 12} értékes jegyet tartalmazzanak!



\noindent
{\bf PéldaInput1}
\verbatiminput{../io/in1}

\noindent
{\bf PéldaOutput1}
\verbatiminput{../io/out1}

\noindent
{\bf PéldaInput2}
\verbatiminput{../io/in7}

\noindent
{\bf PéldaOutput2}
\verbatiminput{../io/out7}


\end{document}
